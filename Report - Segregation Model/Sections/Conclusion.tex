\documentclass[../main.tex]{subfiles}

\begin{document}

For the most part, we consider this project a success. We started by giving a detailed description of Schelling's model of segregation, focusing on the spatial proximity model. We gave a formal definition of convergence and we looked into some of Schelling's intuitions regarding stable final configurations. Our investigation showed that Schelling's turn function is not ideal and we were able to give some examples where the convergence of some initial configurations failed when using Schelling's turn function. In other words, we presented examples of  configurations that do not reach a stable point where no unhappy player is able to improve. However, we followed up Schelling's intuition of relaxing the turn function in order to ensure termination. We managed to rigorously prove that a specific turn function (that was constructed by relaxing Schelling's turn function) will always guarantee a stable final configuration. 

In addition, we successfully implemented a multi-agent computational model using \verb|SMV| model checker. The model was a success because it allowed us to test a variety of linear configurations of different lengths and various neighbourhood sizes. Most importantly, the model was able to generate a Binary Decision Diagram with all the achievable final configurations, considering all the possible orders in which the discontent agents could move. Having all these achievable final configurations allowed us to check how much of an impact the turn function had over the system. We were able to answer questions like "For a given initial configuration and any turn function (unhappy agents are able to move in any order), will the system always converge to a fixed point?" or "Regardless the order in which unhappy agents move, will we always end up with a completed segregated final configuration?". Furthermore, we were able to look at what would be the least segregated final configuration that we could reach. These are all important results since they take us a step closer to understanding if and how we could avoid complete segregation in a system. We are interested in finding the configuration where all the agents are happy and the level of segregation is minimal. 



\subsection{Future work}
In this section, we are going to make some suggestions for possible improvements that can be made in relation to this project:

\begin{itemize}
    \item \textbf{Visualisation Tool}
    
    A tool that allows the user to visualise the \textit{Binary Decision Diagram}(BDD) generated by the \verb|NuSMV| would be a great addition to this project. This way it would be easy to see just about how many final configurations are achievable, how many of those configurations are stable or completely segregated. However, we could see a potential space problem in generated an image of the whole BDD, especially when we talk of thousands or even millions of states. A more feasible tool (something that we have considered creating if we had had more time) would be one that generates a visualisation of the counter-examples that are generated by the model checker. In Appendix \ref{appendix:counter example convergence 3}, we can see an example of a counter-example generated by \verb|NuSMV|. This is not very intuitive and requires a bit of \verb|NuSMV| knowledge to read. Furthermore, a visualisation of the configuration at each step would help the user to acknowledge quicker the direction the counter example is pointing to.  \\
    
    \item \textbf{More Sophisticated Agents}
    
    At the moment, our model is a \textit{two-type agent model}. In other words, in our system, we have only two types of agents: whites or blacks. The colour is the only characteristics that distinguish them, in any other aspects, our agents are the same. A possible extension of the project would be considering more sophisticated agents, taking into account, for example, their economic status, religion, sex, occupation etc. It would be interesting to see in the situation where we have more characteristics to distinguish between the agents if we would end up with more integrated society or, on the contrary, these new characteristics would lead to an even more segregated system.\\
    
    \item \textbf{2D Spatial Proximity Problem}
    
    The system in our project is currently represented by a line where agents (or players) live. The next step would be to model a two-dimension system with black and white agents and possible blank spaces, like in Schelling's model. We would be interested to see how much of an impact the turn function would have in such a configuration.\\
    
    \item \textbf{Solution for State Explosion Problem}
    
    We saw that a model checker has lots of advantages over an object oriented language when designing a multi-agent computational model. The model checker allowed us to be very flexible in terms of the turn functions we tested. The \textit{randomness} of the order in which the unhappy agents could move, allowed us to test more scenarios than we would have if, for example, we would have used strictly Schelling's turn function. However, together with the flexibility in the turn function, comes the large-size system problem. For every new agent, the size of the system grows exponentially and so does the \textit{Binary Decision Diagram} generated by the \verb|NuSMV|. As we mentioned in the previous chapter, this is known as the \textit{State Explosion Problem}. Scientists are working on ways to solve this problem, and one of them could be \textit{Bounded Model Checking}. As an extension for this project, we could implement some of the Bounded Model Checking techniques to check whether certain specifications are met, without having to compile the whole Binary Decision Diagram. 
    
\end{itemize}

%\textbf{Project Plan} \\
%In order to be able to write sophisticated models that could represent real-world situations, we would first have to study and understand simple models, like Schelling's spatial proximity models. First step (which I have already started working on) is using \verb|MATLAB| to test the models and then trying to formally prove a few of the affirmations that Schelling is making. These include a few research questions that we would like to answer, such as:

%\begin{itemize}
%    \item What is a measure of segregation or clustering? (I am currently considering defining it as the average proportion of neighbours of like or opposite colour)
%    \item In which situation (\verb|i.e.| under which initial assumptions) will we get a total segregation? What is total segregation?
%    \item Is it the case that every configuration reaches a fix point, that is that no matter how you move people, there exists a configuration such that no discontent individual is able to improve?
%\end{itemize}

%Once we have finished answering these questions (and maybe a few more that might arise in the future), we would like to move to a slightly more complicated model: bounded-neighbourhood models. This is a more realistic model that we could further use to model neighbourhoods such as schools or even cities.

%Finally, we would like to come up with a new, more sophisticated model that would take into consideration economical and social aspects and try to model some patterns for the economical segregations that is happening in cities like London.

%\textbf{Evaluation Plan}\\
%Being able to formally proof, even just the Schelling's spatial proximity linear models affirmations, would be a great achievement since at this point, although there are quite a few papers talking about these segregation models, there are little to no formal proof regarding them. 

%However, we would still like to move from spatial proximity models to more sophisticated models and we would be really pleased if we could come up with some model patterns for London's economical and ethical segregation.
\end{document}